%%==================================================
%% thanks.tex for SJTU Master Thesis
%% based on CASthesis
%% modified by wei.jianwen@gmail.com
%% version: 0.3a
%% Encoding: UTF-8
%% last update: Dec 5th, 2010
%%==================================================

\begin{thanks}
	
\zihao{4}\hangju{1}
时光荏苒,一晃七年已过。同济见证了一名懵懂少年变成合格硕士的全部历程,
我也见证了同济七年的辉煌。本硕七载青春,留在心里的是道不尽的感激,我
仍将怀着感激之情在母校的怀抱里攻读博士学位以谢母校的哺育之恩。

在这里我首先要感谢我的导师程玖兵教授。从大四至今已四年有余,我的每一
点进步都与您的悉心指导有关。您不仅传授我专业知识,还在生活中给予我巨大
的帮助。从概念到理论,从编程到撰文,您在科研中的每一处细节都用精益求精的
态度来要求自己和我们。每一次的讨论都让我茅塞顿开,每一次的谈话都让我
备受鼓舞。不仅如此,您从不懈怠的工作态度,助人为乐的崇高品格都在
潜移默化的影响着我。如今硕士论文已将完成,其中每一页里面都包含了您的心血
与付出。恩师之情,不轻于父母,无以为报。很荣幸能继续聆听您的教诲,
我将更加努力向您学习。

地震组大家庭是我梦想起航的地方,在这里付出了艰辛与努力,收获了知识、技能
与成长。在这里特别感谢董良国教授这几年来的关心与指导,最初是您的提点让我走上
了科研的道路。同时也感谢耿建华教授在储层地球物理课以及组会上的悉心指导,感谢
王华忠教授在成像与反演以及信号处理中的耐心研讨,感谢宋海滨教授填补的地震海洋学
方面的知识空缺,感谢杨锴教授在地震数据处理
课中的细心讲解,感谢钟广法和刘堂晏教授在地震解释、测井以及岩石物理方面的尽心
辅导,感谢刘玉柱教授在层析成像中的指点迷津,感谢赵峦啸、屠宁和王本锋老师在
组会讨论中的指点和帮助。再次由衷感谢各位老师的传道育人,让我们这些学子受益终身。

同门之谊令我终生难忘。特别感谢王腾飞、王晨龙两位博士师兄对我学习和计算机编程上的
耐心指导以及出差时的细心照顾。感谢已毕业的四位硕士师兄滕龙、康玮、段鹏飞、杨涛
以及尚颖霞师姐留下的丰厚研究成果,让我收益良多。感谢同组的杨亚丽、阮晟、徐文才、
刘学义、于洋、阮轮、熊一能在生活和学业上的无私帮助。此外,感谢一众博士师兄姐王雄文、
李晓波、王义、王毓伟、黄超、迟本鑫、杨积忠、于鹏飞、孙敏傲、张博、杨靖康,你们在
科研上的讨论使我少走了许多弯路。感谢何文俊、王毕文、刘伟刚、康昊的手足之情以及
同届同学三年的同窗之情。

最后,我要感谢父母的养育之恩、弟弟的手足之情,感谢所有亲朋好友在我读书期间
的支持。我特别感谢女朋友饶竹红女士,九年同窗、七年陪伴之情。此生有幸,在最美
的年华遇见你,夫复何求,良辰美景,愿此生与你共度。


\end{thanks}
