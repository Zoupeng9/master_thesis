%%==================================================
%% conclusion.tex for SJTU Master Thesis
%% based on CASthesis
%% modified by wei.jianwen@gmail.com
%% version: 0.3a
%% Encoding: UTF-8
%% last update: Dec 5th, 2010
%%==================================================

\chapter*{结论与展望\markboth{结论与展望}{}}
\addcontentsline{toc}{chapter}{结论与展望}

\newcommand{\backsection}[1]{\vskip 3.8ex
	\begin{flushleft}
		{\zihao{-3}\textbf{#1}}
	\end{flushleft}
	\vskip 1.2ex}


\backsection{结论}

地下介质普遍存在非弹性衰减,它对地震波传播、成像、弹性参数反演以及储层预测均有强烈
影响。本文第二章较系统地总结了地震衰减的数学物理描述,并比较了两类常用粘声
波方程的特性以及在成像、反演中的应用。
吸收衰减对地震波传播的宏观影响是一种沿波路径的累加效应。为了重构地下介质中深部
的衰减模型,本文利用反射波数据来增加波路径的照明深度,首次将$Q$反演纳入
RWI框架。在速度准确的情况下,$Q$-RWI能很好地恢复地下中深部宏观衰减模型。论文最后
通过引入峰值频率移动目标函数来降低$Q$-RWI对高波数速度模型的依赖性。通过论文
的研究,主要得到了以下几点认识:

(1)在勘探地震频段,SLS粘声波方程和常$Q$模型粘声方程均能很好描述地震波在衰减地层中的能量损失
和速度频散效应。在进行成像和反演中既要进行能量补偿同时也必需要保证速度频散关系不变。
常$Q$模型对应的一些近似波动方程可以将能量损失项和速度频散项分开,在进行一次偏移成像时,
用其构造的反传方程能在补偿能量的同时保持速度频散关系不变。而SLS粘声波方程通过记忆变量
来控制能量损失和速度频散,很难用其构造将能量补偿和速度频散分开的反传方程。
两类方程的伴随方程虽对能量进行了二次衰减,但保持了原有的速度频散关系。在进行
迭代反演(如$Q$-LSRTM)时,这两类方程有相同的效果,因为多次迭代能补偿损失的能量。
在进行$Q$反演时,两类方程均可用,但要注意在进行反传时必需用其对应的伴随方程以保证物理关系
的正确性。

(2)在低波数和高波速速度成分均已知的情况下,$Q$-RWI利用反射波数据对中深部
模型进行照明,可以很好恢复中深部宏观$Q$模型。因为反射数据对浅层覆盖不足,$Q$-RWI
对浅层模型的分辨率不高且存在多解性,其浅层反演结果是一种等效模型。这种等效结果虽不能
满足储层预测需求,但足以满足中深层反射波数据$Q$-RTM的需要。

(3)地震数据的峰值频率属性主要与地震本征衰减有关。通过引入峰值频率移动目标函数,很大
程度降低了$Q$-RWI对高波数速度的依赖性。峰值频率对高波数速度的大小和极性不敏感,但强烈
依赖于高波速模型的相对几何结构,即Born正演中的高波数模型不能改变子波的形状。数值实验
证明用层状速度平滑差的结构作为Born正演中高波数扰动模型是可行的。
在实际勘探中,常规处理流程可获取光滑背景速度,再通过偏移成像获得地下反射界面位置,
然后将两层模型平滑差的结构赋在反射界面处以获取高波数速度模型。

\backsection{创新点}

本论文的主要创新点有:

(1)首次将$Q$反演纳入RWI框架,用粘声波动方程作为引擎计算反射波路径以适应复杂地层
情况,通过反射波路径恢复中深部的衰减模型,其反演结果满足$Q$-RTM与$Q$-LSRTM
的需求。

(2)通过将最小化峰值频率移动作为目标函数,降低了$Q$-RWI对高波数速度的依赖性。
通过用层状速度平滑差的结构作为Born正演中的高波数扰动模型,使$Q$-RWI更加满足实际需要。

\backsection{不足与展望}

$Q$-RWI反演仍然存在很多尚待深入研究的问题。作者认为今后还应在以下几方面
继续研究:

(1)当背景速度不准确时,偏移再反偏移的数据与原始数据存在差异,此时$Q$-RWI由于数据匹配不准而
产生误差。如何克服背景速度误差的影响值得研究。

(2)本文是在顺序反演的框架下进行研究的,即先反演速度再反演$Q$值。在实际反演中,地震衰减的存在
也影响速度反演的精度,如何联合反演背景速度和$Q$模型也是非常有意义的研究课题。

(3)本文未进行实际地震数据测试。在实际数据中噪音、薄层调谐、子波形状如何影响峰值频率也是一个值得
研究的点。

(4)此外,除了地层本征衰减,强烈的介质非均匀性造成的散射也可能引起地震波振幅衰减与相位畸变。
因此,在实际应用中,有必要分析散射衰减是不是可以忽略。
