%%==========================
%% chapter01.tex for TJU Master Thesis
%% based on CASthesis
%% modified by charlie.yaha@gmail.com
%% version: 0.1alpha
%% Encoding: UTF-8
%% last update: Dec 5th, 2010
%%==================================================

%\bibliographystyle{TJU} %[此处用于每章都生产参考文献]


\newcommand{\citeA}[2]{\citeauthor{#1}\cite{#1}}

\chapter{绪~论}

\section{研究背景及意义}
地下介质普遍具有衰减性,其衰减性通常用品质因子$Q$来表征。地震衰减对地表反射地震数据有非常重要的影响,
主要表现在如下两方面。首先,地震衰减会减弱地震波的振幅从而减低地震成像的分辨率;其次,地震衰减会使
地震波的相位变形以及速度频散,以致于地震成像不聚焦、位置错位。图(\ref{fig:spectral}a)对比了在同一时刻接收的
非衰减的地震波(蓝色)和在$Q=30$的介质中传播的地震波的波形。从图中可以看出,衰减介质中传播地震波
的振幅被极大程度的吸收衰减。由于地震衰减导致速度频散,高频成分的地震波具有更快的传播速度,所以
两列波具有不同的到达时。同时地震衰减会改造地震波的相位,从而破坏地震波波形的对称性。
图(\ref{fig:spectral}b)对应地展示了图(\ref{fig:spectral}a)中两种波的振幅谱。
从图中可以明显看出,高频成分的衰减要
远远严重于低频成分,这也是造成图(\ref{fig:spectral}a)中振幅强衰减的主要原因。

\begin{figure*}[!htbp]
        \centering
        \includegraphics[width=0.9\linewidth]{figure/spectral}
        \fcaption{地震波在声介质和粘声介质($Q=30$)中传播的数值模拟实验:
	(a)地震波波形图;(b)对应的振幅谱。}{1D example to numerically illustrate the attenuation 
	impacts on the amplitudes spectra and phase of propagating wave: (a) the 
	non-attenuated waveform (blue curve) and the attenuated waveform (red curve); (b) the 
	corresponding amplitude spectral. }[展示地震衰减效应的一维数值实验]
        \label{fig:spectral}
\end{figure*}

在地震波偏移成像中,如果不补偿$Q$的效应,同样会影响地震成像的质量。地震衰减通过吸收地震波的
高频成分和改造地震波的相位来降低地震波成像的质量。图(\ref{fig:qrtm}c)展示了用准确$Q$补偿的
逆时偏移($Q$-RTM)成像结果。像被精确地成在了深度为800m的地方,并且像在强衰减区域下面有
均衡的能量。图(\ref{fig:qrtm}d)是没用$Q$补偿的RTM结果。由于频散的速度在成像中
没有得到校正,成像的位置要稍微浅于800m,并且在强衰减区域下方成像能量弱。在地震勘探过程,
不考虑$Q$的影响可能会导致地震解释不准确,甚至导致钻井错位。

\begin{figure*}[!htbp]
        \centering
        \includegraphics[width=0.9\linewidth]{figure/rtm_no.pdf}
		\fcaption{衰减对成像的影响实验:(a) 速度模型; (b)$Q$模型;(c)$Q$-RTM成像结果;
		(d)RTM成像结果}{ (a) The velocity model. (b) The $Q$ model.
		(c) The migrated image with a correct $Q$ compensation. (d) The migrated image without
		$Q$ compensation.}[衰减对成像的影响实验]
        \label{fig:qrtm}
\end{figure*}

在实际勘探中,气云/气包区的成像、储层识别和解释都面临巨大的挑战。气云/包区通常包含极低的
$Q$值,这种强烈的衰减会吸收深部同相轴的能量,在储层的上方造成成像阴影区,严重影响地震
解释的准确性。可靠的$Q$模型不仅可以提高成像的质量,而且可以更好解释振幅随偏移距变化
(AVO)和各向异性这两种依赖于偏移距的效应。正确的AVO和各向异性
解释可以提高油气勘探的成功率。另外,$Q$模型可以作为一个表征岩石和流体属性的参数,
例如在稀疏井控制下,可以用$Q$模型来检测岩性的边界(\citeA{desgupta.clark:1998},)。
衰减量级是一个直接刻画储层的油气物理参数,例如可以直接利用$Q$模型来确定储层含气/油
的饱和度(\citeA{winkler.nur:1982},)。在油气开发过程中,衰减模型还可以用来指示储层裂缝的方位
(\citeA{maultzsch.chapman:2007}, ,\citeA{clark.benson:2009},)以及监控流体的运移能力,
帮助优化注水过程(\citeA{macrides.kanasewich:1987},)。
因此,在油气勘探开发过程中,定量地评估衰减效应,构建一个可靠的衰减($Q$)模型是非常重要的。
本文主要的工作就是通过反射全波形反演来定量估计地震本征衰减$Q$模型。

\vspace{0.5cm}
\section{国内外研究现状}

国内外学者对$Q$模型估计做了大量的研究工作,其中大部分工作都是在数据域完成。数据域的
反演方法可大致分为两类:基于高频近似的射线层析类和基于波动方程的波形反演类。

射线类方法中,\citeA{brzostowski.mcmechan:1992},首先用观测数据振幅与震源振幅比值的对数作为输入数据来
实现$Q$层析成像。但是除了吸收衰减外,影响振幅的因素还有很多如几何扩散、透射/反射损失、
散射损失等。为了区分衰减引起的振幅损失,\citeA{quan.harris:1997},将观测数据和计算数据间质心
频率的移动作为匹配准则,用射线层析方程来更新$Q$模型。\citeA{hu.liu:2011},用震源的振幅谱作为
拟合函数来处理地震数据频谱非对称性的影响,并用多指数盒状约束法来消除非地震本征衰减的影响。
质心频率移动类的方法相对于振幅匹配类方法对噪音不敏感,所以更适合于处理实际数据。
射线类方法计算效率高,处理简单横向变化的地质构造有很好的效果。但是在上覆介质复杂时,
地震波存在多路径,射线类方法由于不能处理多路径情况而造成误差。波动类层析方法可以
有效的解决多路径问题。

波形反演(FWI)是一种通过求解波动方程来恢复地下介质参数的反演迭代方法(\citeA{tarantola:1984},)。
尽管FWI是一个高度非线性的反演问题(\citeA{mora:1987}, ,\citeA{sirgue.pratt:2004}),由于需要
大量的正演计算,全局寻优的解法(\citeA{sen.stoffa:1991}, ,\citeA{mosegaard.tarantola:1995},)
所需的计算量仍是现在计算机所不能承受的。伴随状态法的引入(\citeA{lailly:1983}, ,
\citeA{tarantola:1984}, ,\citeA{pratt.worthington:1990},),使得梯度的计算变得非常高效,目前基于
梯度类的解法已趋于成熟。在勘探地震学中,各种应用实例证明粘声波形反演($Q$-FWI)对于提供高
分辨率的P波速度结构有非常重要的作用(\citeA{song:1995}, ,\citeA{ravaut:2004}, 
,\citeA{gao:2006}, , \citeA{kamei:2012},)。但是衰减的波形反演比速度的波形反演更具挑战性(
\citeA{song:1995}, ,\citeA{liao.mcmechan:1996}, ,\citeA{smithyman:2009}, ,\citeA{hak.mulder:2011}, 
,\citeA{malinowski:2011}, ,\citeA{bai:2014}, ,\citeA{kamei.pratt:2013},)。

考虑衰减参数的波形反演第一次出现在\citeA{tarantola:1988},的时间域粘弹波形反演理论中。
随后,\citeA{song:1995}在频率域提出了一种粘声波形反演方法。在频率域实现波形反演比时间域
有如下优势,尤其是考虑衰减效应: 首先,衰减参数(例如$Q$值)和频散速度关系很容易用随频率
变化的复速度来表示,速度和衰减参数的梯度可以同时得到而不需要额外的计算量(\citeA{song:1995},)。
另外,频率域反演方法可以自然的实现多尺度反演策略以降低波形反演的非线性(\citeA{bunks:1995}, 
,\citeA{sirgue.pratt:2004},)。从最低的频率成分开始反演,并逐渐逐渐增加高频成分,这样可以
逐渐恢复地下模型的波数成分。

\vspace{0.5cm}
\section{本文研究内容}


















