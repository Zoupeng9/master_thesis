%%==================================================
%% chapter03.tex for TJU Master Thesis
%% Encoding: UTF-8
%%==================================================

\chapter{基于粘声方程的反射全波形反演}

对于目前普遍缺乏可靠的低频(小于4Hz)以及长偏移距信息的地震数据,中深层的
参数信息主要包含在反射数据中,常规的FWI对于中深层参数的建模往往无能为力。
因此,反射波波形反演(RWI)成为近几年国际地震勘探领域争相研究的课题。沿
反射波波路径进行背景速度反演已经取得了很多成功的案例。

RWI是处理FWI的强非线性问题的一种有效解决途径,虽然在背景速度建模中被广泛
使用,但是目前为止,尚未有学者将其引入到背景$Q$的估计中。地震衰减对地震波传播
的影响相较于速度对地震波传播的影响既有相似之处,也略有不同。本章将详细介绍
如何把RWI引入到背景$Q$的反演中。

\vspace{0.5cm}
\section{引言}
\vspace{0.5cm}

用地震波信息来反演地下地层的衰减信息已有很多年的研究历史,其反演方法可大致
分为射线层析和波形反演两类。射线层析类方法(\citeA{brzostowski.mcmechan:1992};
\citeA{quan.harris:1997};\citeA{hu.liu:2011})用射线路径来构建层析矩阵,具有很高的
计算效率,对于介质变化简单的情况有很好的应用效果。但当介质复杂时,地震波传播往往
存在多路径,并且当介质存在高速或低速透镜体时,射线追踪会出现焦散现象,不能对其下覆
地层进行很好的照明。波动类的反演方法以全波动方程为传播引擎,能较精确的刻画地震波
在复杂介质中传播的各种波现象,理论上是目前精度最高的参数建模方法。自从
\citeB{lailly:1983}和\citeB{tarantola:1984}等人建立起全波形反演的基本理论框架以来,
越来越多的人关注该方法的研究与应用,尤其是在地震波速度反演方面。在地震品质因子
反演方面,\citeB{tarantola:1988}第一次提出了时间域粘弹波形反演理论。随后,
\citeB{song:1995}在频率域提出了一种粘声波形反演方法($Q$-FWI)。在$Q$-FWI中,速度
参数和品质因子有很强的耦合性(\citeA{song:1995};\citeA{kamei.pratt:2008};
\citeA{hak.mulder:2011})。在观测系统不完备的情况下,解决这中耦合性的思路有
两种,一种是通过预条件模型参数的梯度(\citeA{liao.mcmechan:1996};
\citeA{hak.mulder:2011};\citeA{malinowski:2011})来减弱耦合。另一种是顺序
反演法,先反演对地震数据影响较强的速度参数然后再反演$Q$模型(\citeA{pratt:2004};
\citeA{rao.wang:2008};\citeA{smithyman:2009})。

$Q$-FWI的成功需要低频长偏移距地震数据,但是这种高质量的数据采集需要高昂的采集费用。
当缺少长偏移距的折射数据时,深部模型往往只被反射波照明。因此,利用反射波进行中
深层的速度建模已成为地震成像的共识(\citeA{xu:2012a};\citeA{ma.hale:2013};
\citeA{chi:2015})。但是目前尚未有人用反射波波路径来进行地震衰减建模。本章
在顺序反演的思路下,先假设已获得准确的速度模型,然后引入RWI框架,用反射波来
反演背景$Q$模型。

\vspace{0.5cm}
\section{方法原理}
\vspace{0.5cm}
当利用反射波进行反演时,常规FWI对模型的短波长结构更敏感,而地震衰减对地震数据的
影响是一种沿地震波波路径的累加效应,因而,其长波长的背景模型显得更为重要。RWI通过
将数据残差投影到反射波波路径来构建模型的长波长成分,更加切合对地震衰减参数反演的
需求。本节将介绍粘声反射全波形反演($Q$-RWI)方法原理。

\vspace{0.5cm}
\subsection{反射波波形反演基本思路}
\vspace{0.5cm}

\vspace{0.5cm}
\subsection{基于粘声方程的反射全波形反演}
\vspace{0.5cm}

\section{数值实验}

