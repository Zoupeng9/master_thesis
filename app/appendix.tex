
\appendix	% 使用英文字母对附录编号,重新定义附录中的公式、图图表编号样式
\renewcommand\theequation{\Alph{chapter}--\arabic{equation}}	
\renewcommand\thefigure{\Alph{chapter}--\arabic{figure}}
\renewcommand\thetable{\Alph{chapter}--\arabic{table}}

\chapter{低秩近似求解DCQ方程}
\citeA{zhu.harris:2014},推导的DCQ方程有如下的频散关系:
\begin{equation}
    \frac{\omega^2}{c^2}=-\eta|\mathbf{k}|^{2\gamma+2}-i\omega\tau|\mathbf{k}|^{2\gamma+1},
	\label{eq:disp}
\end{equation}
求解方程(\ref{eq:disp})的$\omega$有,
\begin{equation}
	\omega=\frac{-ip_1+p_2}{2},
\end{equation}
其中
\begin{equation}
	p_1=\tau c^2|\mathbf{k}|^{2\gamma+1},
\end{equation}
\begin{equation}
	p_2=\sqrt{-\tau^2c^4|\mathbf{k}|^{4\gamma+2}-4\eta c^2|\mathbf{k}|^{2\gamma+2}}.
\end{equation}
在时间域决定波传播的相移函数可以定义为:
\begin{equation}
	\phi_1(\mathbf{x},\mathbf{k},\Delta t)=\mathbf{k}\cdot\mathbf{x}+\frac{-ip_1+p_2}{2}\Delta t,
	\label{eq:phase_func}
\end{equation}

在$Q$-RTM中,通过改变方程(\ref{eq:disp})右端第二项的符号来补偿振幅衰减,而频散控制项的符号
必须保持不变。因此,衰减补偿常$Q$方程对应的频散关系为:
\begin{equation}
    \frac{\omega^2}{c^2}=-\eta|\mathbf{k}|^{2\gamma+2}+i\omega\tau|\mathbf{k}|^{2\gamma+1},
	\label{eq:disp1}
\end{equation}
其对应的复共轭相位函数为:
\begin{equation}
	\phi_2(\mathbf{x},\mathbf{k},\Delta t)=\bar{\phi_1}(\mathbf{x},\mathbf{k},\Delta t)
    =\mathbf{k}\cdot\mathbf{x}+\frac{ip_1+p_2}{2}\Delta t,
	\label{eq:phase_func1}
\end{equation}
相位函数$\phi_1$和$\phi_2$都包含波数的分数阶指数,同时是空间位置$\mathbf{x}$和波数$\mathbf{k}$
的函数。\citeA{sun.zhu:2015},用这种相位函数来延拓粘声波场,并允许分数阶指数$\gamma(\mathbf{x})$
随空间变化。单步混合域算子可以整合成如下的Fourier积分算子:
\begin{equation}
	p(\mathbf{x},t+\Delta t)=\int \hat{p}(\mathbf{k},t)e^{i\phi(\mathbf{x},\mathbf{k},\Delta t)}
	d\mathbf{k},
	\label{eq:one_step}
\end{equation}
其中$\hat{p}$是$p$的空间Fourier变换。方程(\ref{eq:one_step})的伴随形式为:
\begin{equation}
	\hat{p}(\mathbf{k},t)=\int p(\mathbf{x},t+\Delta t)e^{-i\bar{\phi}(\mathbf{x},\mathbf{k},
	\Delta t)}d\mathbf{k},
\end{equation}
式中$\bar{\phi}$是$\phi$的共轭。
将方程(\ref{eq:phase_func})带入方程(\ref{eq:one_step})中可以显示得到粘声方程的正传算子
\begin{equation}
	\begin{aligned}
		p(\mathbf{x},t+\Delta t)&=\int \hat{p}(\mathbf{k},t)e^{i\phi_1(\mathbf{x},
		\mathbf{k},\Delta t)}d\mathbf{k}, \\
		&=\int \hat{p}(\mathbf{k},t)e^{i\mathbf{k}\cdot\mathbf{x}+(p_1+ip_2)\Delta t/2}d\mathbf{k}.
	\end{aligned}
	\label{eq:forward}
\end{equation}
其对应的伴随算子为:
\begin{equation}
	\begin{aligned}
		\hat{p}_{adj}(\mathbf{k},t)&=\int p(\mathbf{x},t+\Delta t)e^{-i\bar{\phi}_1(\mathbf{x},
		\mathbf{k},\Delta t)}d\mathbf{k}, \\
		&=\int \hat{p}(\mathbf{x},t+\Delta t)e^{-i\mathbf{k}\cdot\mathbf{x}+(p_1-ip_2)\Delta t/2}d\mathbf{k}.
	\end{aligned}
	\label{eq:adjoint}
\end{equation}
伴随算子(\ref{eq:adjoint})作为反传算子补偿了$Q$-RTM中的速度频散,但是会二次衰减波场的能量,所以不适合
$Q$-RTM。
在$Q$-RTM中,正传和反传波场都应该补偿能量的衰减,其对应的正传和反传算子如下:
\begin{equation}
	\begin{aligned}
		p_{comp}(\mathbf{x},t+\Delta t)&=\int \hat{p}(\mathbf{k},t)e^{i\phi_2(\mathbf{x},
		\mathbf{k},\Delta t)}d\mathbf{k}, \\
		&=\int \hat{p}(\mathbf{k},t)e^{i\mathbf{k}\cdot\mathbf{x}+(-p_1+ip_2)\Delta t/2}d\mathbf{k},
	\end{aligned}
	\label{eq:forward1}
\end{equation}
和
\begin{equation}
	\begin{aligned}
		\hat{p}_{comp-adj}(\mathbf{k},t)&=\int p(\mathbf{x},t+\Delta t)e^{-i\bar{\phi}_2(\mathbf{x},
		\mathbf{k},\Delta t)}d\mathbf{k}, \\
		&=\int \hat{p}(\mathbf{x},t+\Delta t)e^{-i\mathbf{k}\cdot\mathbf{x}+(-p_1-ip_2)\Delta t/2}d\mathbf{k}.
	\end{aligned}
	\label{eq:adjoint1}
\end{equation}

以上方程中的混合域算子$e^{i\phi(\mathbf{x},\mathbf{k},\Delta t)}$统一
用$W(\mathbf{x},\mathbf{k})$来表示,由于
其涉及空间-波数域的相互变换,导致数值计算非常昂贵。\citeA{fomel.ying:2013},用低秩近似来降低
混合域算子的计算量。其形式可以表示为如下形式:
\begin{equation}
	W(\mathbf{x},\mathbf{k})\approx\sum_{m=1}^M\sum_{n=1}^NW(\mathbf{x},\mathbf{k}_m)a_{mn}
	W(\mathbf{x}_n,\mathbf{k})
\end{equation}
因此$p(\mathbf{x},t+\Delta t)$的计算有如下的形式:
\begin{equation} 
	p(\mathbf{x},t+\Delta t)\approx\sum_{m=1}^MW(\mathbf{x},\mathbf{k}_m)\(\sum_{n=1}^Na_{mn}\(\int
	e^{i\mathbf{x}\mathbf{k}}W(\mathbf{x}_n,\mathbf{k})p(\mathbf{k},t)d\mathbf{k}\)\).
	\label{eq:lowrank}
\end{equation}
方程(\ref{eq:lowrank})的计算量相当于每个时间步做$N$次反Fourier变换,在实际中,$N<10$。


\chapter{DCQ方程反射全波形反演梯度推导}
