%%==================================================
%% abstract.tex for TJU Master Thesis
%% based on CASthesis
%% version: 0.3a
%% Encoding: UTF-8
%% last update: Dec 5th, 2010
%%==================================================


%中文摘要

\begin{abstract}

实际地球介质是非弹性的,当地震波在地层中传播时将引起地震衰减,衰减强度通常
用品质因子$Q$来描述。强地震衰减往往出现在油气勘探的目标区域如汽云、
汽包区。地震衰减对地震波的运动学和动力学特征都会产生影响:速度频散和振幅衰减。
在地震数据处理中,如果不考虑地震衰减效应将直接影响储层识别和解释的可靠性。
在偏移成像中可以通过衰减补偿来校正速度频散和能量损失。然而所有的带
衰减补偿的偏移算法均需要相对准确的背景$Q$模型。

$Q$模型估计一直是地震学和勘探地震学领域的研究热点之一。早期用衰减前后地震波
的振幅谱的比值来构建层析矩阵,为了消除非衰减因素对振幅谱的影响,学者们通过检测
质心频率的变化来反演$Q$值。这类方法通过射线追踪计算层析矩阵,当介质变化复杂时
射线理论存在焦散以及多路径效应,从而降低了此类方法反演的精度。
随着高性能计算机技术的不断进步以及宽频带、宽方位、高密度地震数据采集技术的逐渐
成熟,单参数速度的全波形反演(FWI)技术逐渐走向成熟。近年来,粘声介质的全波形
反演($Q$-FWI)技术也得到了飞速发展。$Q$-FWI以波动方程为传播引擎,能适应复杂地质
情况,但其利用的主要是透射波信息,透射波对地下的穿透深度有限。当缺乏低频、长
偏移距数据时,$Q$-FWI只能反演得到浅层的模型参数。为了获得中深部的$Q$模型,
在成像域中提出了偏移$Q$分析方法。但成像域方法相较于数据域方法其分辨率偏低。

本文第3章首次将$Q$反演纳入反射全波形反演($Q$-RWI)框架。当观测地震数据缺少大孔径的
折射波信息时,$Q$-RWI利用反射波数据更新中深层模型。$Q$-RWI强烈依赖高波数速度
模型,在缺乏$Q$信息的情况下,最小平方逆时偏移(LSRTM)没法给出准确的高波数
速度模型。为了降低$Q$-RWI对速度模型高波数成分的依赖性,论文第4章引入了峰值频率
移动目标函数来更新中深层$Q$模型。峰值频率是地震数据的一个属性,其大小只与地震子波
以及地层的衰减有关,只要速度的高波数成分不影响地震子波的形状即不会改变峰值频率的大小。
在实际反演中用层状速度平滑差代替高波数速度扰动即可满足要求。

本文针对中深层背景$Q$反演遇到的问题,提出了$Q$-RWI反演方法。通过峰值频率移动
目标函数的引入全完降低了$Q$-RWI对高波数速度模型的依赖,使$Q$-RWI方法更加
实用化。




  \keywords{\zihao{-4} 
  	地震衰减,  
  	全波形反演, 
  	粘声反射全波形反演,  
  	峰值频率,  
  	衰减补偿逆时偏移 
  	}
\end{abstract}



%英文摘要


\begin{englishabstract}

The real earth model is anelastic, which cause seismic attenuation and the 
anelastic effects are described by the quality factor $Q$. Strong subsurface attenuation 
mainly occur at the target area of oil exploration such as gas clouds/pockets. 
Seismic attenuation will lead amplitude decrease and phase distortion as seismic wave 
propagating in the earth. It will decrease the reliability of reservoir identification 
and interpretation if we don't take into account the seismic attenuation when process the seismic 
data. The amplitude losses and the dispersion from attenuation are often compensated during 
seismic migration. However, most migration algorithms with attenuation compensation 
require an estimate of background $Q$ model.

The background $Q$ model estimation is one of the research hotspots in the fields of 
seismology and exploration geophysics. The researchers first use the log of the recorded 
amplitudes and the source amplitudes as the input data for the ray-based $Q$ tomography.
In order to distinguish the amplitudes loss from the elastic effects, the centroid-frequency 
shifts between the predicted and the observed traces are smeared along raypaths to update 
the $Q$ model. These methods rely on the high frequency assumption made in classical 
ray-based tomography. In complex media, the application of those
methods in building a reliable $Q$ model is a well-known challenge because the limitations 
of the ray methods compromise the finite frequency nature of seismic wave propagation.
With the development of high-performance computer and the better quality of broadband, full 
aperture and high density seismic data, the full waveform inversion (FWI) technique for 
one single velocity parameter is more and more sophisticated. There are also many published 
studies use FWI to update the $Q$ ($Q$-FWI) model. $Q$-FWI use the wave equation to 
depict the wavepath, which provides a superior way of building high-resolution models, 
especially under complex geological setting, but often suffers from an objective 
function with local minima caused mainly by the absense of low frequencies and long-offsets 
in seismograms. In order to gain the deep part backgroud $Q$ model, the wave-equation 
migration $Q$ analysis method was proposed in the image domain. However, the inversion 
results from image domain has lower resolution than data domain.

In the chapter 3, we developed a visco-acoustic wave equation reflection waveform 
inversion ($Q$-RWI) method to update the deep part of background $Q$ model when 
miss the long-offsets and low frequencies seismgrams. However, the $Q$-RWI strongly 
rely on the high wavenumber component of the velocity model. What's more, 
the least square reverse time migration (LSRTM) can't obtain a reliable perturbation 
velocity model when the $Q$ model is unkonwn. In the chapter 4, we use peak frequency 
shifts between the the predicted and the observed traces as the misfit function 
to descrease the dependency of the perturbation velocity model. The peak frequency 
is one of attributes of the seismic traces and only depends on the seismic wavelet 
and seismic attenuation. In Born forward modeling, the peak frequency could not be
changed as long as the pertubation velocity model couldn't chage the shape of the 
wavelet. In practice, we use the residual model between layered velocity model and it's 
smoothing model to replace the high wavenumber velocity perturbation. 

In this thesis, we developed a $Q$-RWI method to update the deep part of $Q$ model.
We decrease the dependency of $Q$-RWI to high wavenuber velocity model through introducing 
peak frequencies shifts misfit function and make it more practical.


  \englishkeywords{\zihao{-4}  
  	Seismic attenuation, \quad 
  	Full waveform inversion, \quad
  	Visco-acoustic reflection waveform inversion, \quad
	Peak frequancy, \quad
	Attenuation compensation reverse time migration
  }
  
  
\end{englishabstract}
